\section{Diskussion}
\label{sec:Diskussion}
Es lassen sich herausgefundene Temperaturen vergleichen.
Die über das Anlaufstromgebiet herausgefundene Temperatur (Unterkapitel \ref{subsec:c}) war zu hoch.
So hat die Temperatur in Unterkapitel \ref{subsec:d} bei den gleichen Heizeigenschaften zu dem Wert eine relative Abweichung von ca. 40\%. Der höhere Wert
überschreitet bereits den Schmelzpunkt (3680 $\pm$ 20 K \cite{wolfram}) von Wolfram und ist daher höchst unwahrscheinlich. Der Fehler kommt wahrscheinlich zu stande, da die Ströme in dem Anlaufstromgebiet sehr klein sind und
die verwendeten Messgeräte dementsprechend sehr empfindlich und anfällig für Fehler sind.
Der Wert der Austrittsarbeit war im Durchschnitt nah am Literaturwert (4.55 eV \cite{wolfram}) mit einer relativen Abweichung von lediglich 2.20\%.
Um die Ergebnisse genauer zu bekommen, wäre es zu empfehlen, die Messungen jeweils mehrmals durchzuführen, vor Allem für das Anlaufstromgebiet.
So könnte man Ungenauigkeiten besser eliminieren.