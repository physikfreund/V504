\section{Diskussion}
\label{sec:Diskussion}

Der Exponent des Fits bei \label{eq:kenn4ergebnis} liegt bei 1.407 und hat somit eine relative Abweichung von 6.2 \% von dem angegebenen Wert 1.5.

Die über das Anlaufstromgebiet herausgefundene Temperatur (Unterkapitel \ref{subsec:c}) war zu hoch.
So hat die Temperatur in Unterkapitel \ref{subsec:d} bei den gleichen Heizeigenschaften zu dem Wert eine relative Abweichung von ca. 40\%. Der höhere Wert
überschreitet bereits den Schmelzpunkt (3680 $\pm$ 20 K \cite{wolfram}) von Wolfram und ist daher höchst unwahrscheinlich. Der Fehler kommt wahrscheinlich zu stande, da die Ströme in dem Anlaufstromgebiet sehr klein sind und
die verwendeten Messgeräte dementsprechend sehr empfindlich und anfällig für Fehler sind. 

Folgende Tabelle gibt die Kathodentemperatur bei dem Versuch und die Austrittsarbeit von Wolfram an.
\begin{table}
 \centering
 \caption{Austrittsarbeiten}
 \label{tab:alles}
 \begin{tabular}{|c c c c |}  
 \toprule
$I_H$ / A & $U_H$ / U & Austrittsarbeit in $eV$ & relative Abweichung der Austrittsarbeit vom Literaturwert & Temperatur in K\\
\midrule
5A & 2V & 3.82 & 16.0 \% & $2048.87$ \\     
5.5A & 2.2V & 3.73 & 18.0 \% & $2159.16$\\
6A & 2.4V & 3.71 &  18.5 \%&$2263.24$\\
\bottomrule
\end{tabular}
\end{table}
Die bestimmten Austrittsarbeiten sind zwar in der gewünschten Größenordnung, haben jedoch eine relativ hohe relative Abweichung von 16 bis 18 \%.